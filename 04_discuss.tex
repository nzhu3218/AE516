\section{Discussion}

Is the agreement between your CFD model and XFOIL within this same tolerance level for lift and drag? (~10\% error bar)
Taking the XFoil data as "experimental," and the CFD data as numerical prediction, error bars of $\pm10\%$ were drawn from each XFoil data point. Observing the $C_l$ first in Fig. \ref{fig:error_bar1}, XFoil and CFD agree very well with the exception of AoA = -2 deg (I need to double check this run), where there is a noticeable deviation. Notice that the point of separation is earlier for CFD predictions. In this case, the CFD simulation might be more trustworthy, as it is difficult to predict separation, especially with low-fidelity methods. Regarding $C_d$, the CFD generally overpredicted the XFoil solution as seen in Figs. \ref{fig:error_bar2} and \ref{fig:error_bar3}. The former is at small negative AoAs, where the CFD prediction is about 10\% above XFoil prediction. This was as close as the two solutions ever got, higher AoAs lead to larger discrepancies, where the CFD prediction increasingly overpredicts the XFoil estimates.

\begin{figure}[H]
\centering
    \includegraphics[width=0.65\textwidth]{error_bar1.jpg}
    \caption{$C_l$ and $C_d$ comparisons including error bar of $\pm$10\%}
    \label{fig:error_bar1}
\end{figure}

\begin{figure}[H]
  \begin{subfigure}[b]{0.5\textwidth}
    \includegraphics[width=\textwidth]{error_bar2.jpg}
    \caption{Zoomed-in view of section of $C_d$}
    \label{fig:error_bar2}
  \end{subfigure}
  \begin{subfigure}[b]{0.5\textwidth}
    \includegraphics[width=\textwidth]{error_bar3.jpg}
    \caption{Another zoomed-in view of section of $C_d$}
    \label{fig:error_bar3}
  \end{subfigure}
\end{figure}
